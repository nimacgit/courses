\documentclass[a4paper]{article}

\usepackage{graphicx}
\usepackage{multirow}
\usepackage{amsmath}
\usepackage{enumitem}
\usepackage{blindtext}
\usepackage{listings}
\usepackage{tikz}
\usepackage{hyperref}
\usetikzlibrary{automata,positioning,arrows}

\usepackage{xepersian}
\settextfont{IRANSans}
\setlatintextfont{Lucida Sans}

\title{تمرین دوم دی ای}
\author{نیما بهرنگ 96100114}
\date{\today}
\begin{document}
\maketitle
\begin{center}
استاد مرتضی علیمی
\end{center}

\section*{پرسش ۱}
با استفاده از روش تقسیم و حل مسئله را حل می کنیم. به این صورت که با مرتب کردن داده ها بر اساس محور ایکس ها که یکبار انجام می دهیم و مقدار 
\lr{$nlog(n)$}
زمان نیاز دارد، سپس مقداری که در میانه یا وسط این نقاط است را انتخاب می کنیم، ممکن است تعداد بیش از یک نقطه مقدار انتخاب شده را داشته باشند، ولی فعلا برای ما مقدار ایکس آنها مهم است. پس مقدار ایکس نقطه ای که کمتر از نصف نقاط در سمت راست و کمتر از نصف نقاط در سمت چپ آن است را انتخاب کرده ایم\\
حال مسئله را برای نقاط چپ و راست حل می کنیم\\
حال به ازای نقاط اصلی مسئله که داشته ایم، روی محور ایگرگی که از نقطه انتخاب شده وسطی، می گذرد، نقطه اضافه می کنیم. برای مثال اگر ایکس انتخاب شده اولیه، ۲ باشد، روی محور ایگرگ که از ۲ می گذرد، به تعداد نقاطی که در سمت راست یا چپ آن است، تصویر آن نقاط بر روی این محور را اضافه می کنیم\\
پس به تعداد کمتر مساوی نقاط اولیه مسئله در این گام نقطه روی این محور افزوده ایم. این کار را به صورت بازگشتی انجام می دهیم\\
\lr{$T(n) = 2T(\dfrac{n}{2}) + O(n)$}\\
که این فرمول هم برای مرتبه زمانی و هم مرتبه نقاط برقرار است زیرا باید در هر گام به ازای هر نقطه در مسئله اولیه، نقطه جدید روی آن محور بیافزاییم.\\
که طبق قضیه مستر این روش بازگشتی اوردرش برابر
\lr{$O(n logn)$}
است\\
حال اثبات می کنیم که شروط مسئله را دارد\\
طبق فرض استقرا هر سمت این شرط را دارد و به ازای حالت های یک و هیچ نقطه نیز بدیهی برقرار است پس کافی است نشان دهیم که به ازای مرج کردن جواب های هر سمت نیز، شروط برقرار است\\
هر دو نقطه روی محور که به طور بدیهی شرط برابری ایکس را دارند و هر دو نقطه در هر سمت نیز طبق فرض استقرا یکی از ۳ شرط را دارند\\
جال ۲ حالت می ماند، نقاط یک سمت، با نقاط روی محور و حالت دیگر نقاط یک سمت با سمت دیگر\\
یک نکته که در این روش وجود دارد این است که ما نقاط جدید را در همان ایگرگ قبلی نقاط پیشین اضافه می کنیم، یعنی اگر نقطه جدیدی اضافه شده است، نقطه ی دیگری وجود داشته که در همان ایگرگ بوده\\
در حالت اول چون هر نقطه ابتدایی را معادلش بر روی محور وجود دارد پس هر نقطه در یکی از سمت ها با هر نقطه روی محور، یا شرط ۲ را دارد یا ۳ زیرا اگر نقطه انتخاب شده در یکی از سمت های محور، از نقاط مسئله اصلی باشد، متناظرش روی محور وجود دارد پس حالت ۲ یا ۳ رخ می دهد بسته به اینکه نقطه انتخاب شده روی محور، ایگرگش با آن نقطه یکی باشد یا نه و اگر نقطه انتخاب شده از نقاط افزوده شده در مراحل باشد، نقطه متناظری هم ایگرگ با آن وجود دارد که عامل بوجود آمدن این نقطه بوده، پس ما یک نقطه به ازایش بر روی محور گذاشته ایم پس باز یک از شرط های ۲ یا ۳ رخ می دهد\\
حال برای نقاط از دو سمت محور نیز همانند استدلال بالا، به ازای هر نقطه، یک نقطه هم ایگرگ با آن وجود دارد پس باز یا حالت۳ رخ می دهد که نقطه ای روی محور وجود دارد که بین این دونقطه است و یا این نقاط هم ایگرگ هستند که حالت ۲ رخ می دهد.\\
پس با افزودن این نقاط شروط مسئله در هر گام برقرار می ماند

\pagebreak

\section*{پرسش ۲}
\begin{enumerate}
\item{}
یک گراف به ازای باقی مانده های صفر تا 
\lr{n-1}
می سازیم که نشان دهده رئوس هستند و از مقدار صفر شروع می کنیم و هر گام یک رقم به اعدامان اضافه می کنیم
\item{}

\end{enumerate}


\pagebreak

\section*{مراجع}




\end{document}
