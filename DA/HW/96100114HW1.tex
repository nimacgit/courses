\documentclass[a4paper]{article}

\usepackage{graphicx}
\usepackage{multirow}
\usepackage{amsmath}
\usepackage{enumitem}
\usepackage{blindtext}
\usepackage{listings}
\usepackage{tikz}
\usepackage{hyperref}
\usetikzlibrary{automata,positioning,arrows}

\usepackage{xepersian}
\settextfont{IRANSans}
\setlatintextfont{Lucida Sans}

\title{تمرین اول دی ای}
\author{نیما بهرنگ 96100114}
\date{\today}
\begin{document}
\maketitle
\begin{center}
استاد مرتضی علیمی
\end{center}

\section*{پرسش ۱}
\begin{enumerate}

\item{}
درست بودن یا نبودن یک کار، وابسته به شرایط و باورهای افراد است\\

وابسته به شرایط ممکن است دو مسیر را انتخاب کنیم، اصیل
\lr{(authentic)}
 یا غیر اصیل برخورد کردن با مسئله.\\
 
 در مسیر اول جواب ها را از کسی گرفته و با تلاش برای حل مجدد سوال می کنیم و یا صرفا کپی کردن به صورتی که تقلب گرفته نشود.\\
 
 مسیر دوم می تواند سه راه داشته باشد. یکی تحویل ندادن تمرین، یکی کمتر خوابیدن و حل تمرین توسط خود و دیگری گرفتن تمرین از دوستی و تلاش برای حل مجدد تمرین ها با راهنمایی و کمک از راه حل های گرفته شده از دوستمان و اعلام صریح اینکه قائده ندیدن تمرین های دیگران را رعایت نکرده ایم اما تمرین ها را بدون تلاش رها نکرده ایم و برای حل کردنشان در حد توان تلاش کرده ایم.\\
 راه سوم به این حرف که "اصالت با اصیل بودن با نااصالتی هایمان شروع می شود" نزدیک تر است\\
 
 \item{}
همانند قسمت قبل به تعهد و باورهایمان وابسته است که در هر شرایط چه تصمیمی می گیریم، این شرایط نیز همانند قسمت ۱، بنا به تصمیمان بر اصالت یا عدم اصالت، انتخاب های گفته شده را داریم و در هر مورد، انتخاب گفته شده با توجه به تصمیمان بر اصالت و عدم اصالت،درست می باشد.
 
\item{}
نگاه کردن به جواب سایت، مجاز نیست اما ارجاع دادن پس از استفاده، لازم است\\

\item{}
طبق قوائد کلاس، ذکر شده که رفع اشکل همکلاس هایمان را نباید بکنیم و در ابتدای متن آمده که تمرین باید کار خود دانشجو باشد که این نیز ابهام را نمی زداید ولی برداشت شخصی بنده این است که چرخ را همیشه از ابتدا خلق کردن، روش موثری نیست ولی می توان با ذکر نام فرد کمک گرفته، اصالت خود را حفظ کرد.\\

 \end{enumerate}
\pagebreak

\section*{پرسش ۲}
ما متعهد به اجرای وعده درس خواندن برای میانترم، پایانترم و تمرینات هستیم\
\
تمامیت(integrity) داشتن در اینجا به معنای این است که به حرفمان عمل کنیم و اولین زمانی که نمی توانیم به حرفمان عمل کنیم، آن را به افراد ذی نفع اطلاع دهیم و به جبران عدم انجام کارمان بر اساس تصمیم ذی نفعان بپردازیم\\
پس گزینه ۴ مصداق اجرای تمامیت است\\



\pagebreak

\section*{پرسش ۳}
\begin{enumerate}
\item{}
درست است زیرا طبق اثبات توسط درخت متوازن دو دویی، هر چیزی که در بدترین حالت کمتر از این اوردر باشد و مبتنی بر مقایسه باشد، حالتی یافت می شود که مقدار بیشتر مساوی
\lr{n lgn}
زمان نیاز داشته باشد
\item{}
بله چون انتخاب ۱۰ راس از n راس از اوردر
 $ n^10$
 است، پس الگوریتم بدیهی این است که تمام  این حالات را چک کنیم که مستقل هستند یا نه که چک کردن شرط استقلال خود به ۵۵ مقایسه نیاز دارد  که همان مقایسه جفت راس ها برای وجود یال است.پس اوردر را تغییر نمی دهد پس بهینه الگوریتم، از این مقدار کوچکتر نیاز به محاسبه دارد پس باز از این اوردر می باشد\\
 
\item{}
استدلال ذکر شده غلط است. یعنی به تنهایی دلیلی برای چک کردن تمام حالات نیست. اساسا الگوریتم بهتر الگوریتمی است که فضای حالت کمتری را چک می کند. مثلا برای مرتب سازی اعداد به جای مقایسه دو به دوی اعداد، مقدار کمتری را چک می کنیم تا به جواب برسیم یعنی درخت حالات ممکن را کمتر کرده ایم. پس با این استدلال این حرف غلط است اما ممکن است حرف مستقلا درست باشد\\

\item{}

هش کردن که الگوریتمی با اوردر تقریبا عدد ثابت است به ما کمک می کند تا در حافظه n اطلاعات را ذخیره و در اوردر ۱ هر کدام را بدست اوریم

\end{enumerate}

\pagebreak

\section*{مراجع}
تجربه حدود ۲۰ سال زندگی خود\\
سمینار راهبری موثر موسسه پویش\\
مقالات راهبری ورنر ارهارد\\
\href{https://papers.ssrn.com/sol3/papers.cfm?abstract_id=3081564}{قایل ارائه راهبری موثر}



\end{document}
