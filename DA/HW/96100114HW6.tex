\documentclass[a4paper]{article}

\usepackage{graphicx}
\usepackage{multirow}
\usepackage{amsmath}
\usepackage{enumitem}
\usepackage{blindtext}
\usepackage{listings}
\usepackage{tikz}
\usepackage{hyperref}
\usetikzlibrary{automata,positioning,arrows}

\usepackage{xepersian}
\settextfont{IRANSans}
\setlatintextfont{Lucida Sans}

\title{تمرین ۶ دی ای}
\author{نیما بهرنگ 96100114}
\date{\today}
\begin{document}
\maketitle
\begin{center}
استاد مرتضی علیمی
\end{center}

\section*{پرسش ۱}
چون درخواست ها را داریم، از انتها به ابتدا مسئله را حل می کنیم. پس یکبار تمام در خواست های حذف را اعمال می کنیم تا گراف باقی مانده در انتها بدست آید. حال با الگوریتم فلوید-وارشال فاصله بین هر زوج را حساب می کنیم. در اینجا از تحلیل سرشکن و مدل خسته(lazy) برای اجرای الگوریتم فلوید-وارشال استفاده می کنیم.\\
فرض می کنیم شماره راس ها به ترتیب اضافه شدن از انتهای درخواست ها است، یعنی راسی که اولین بار حذف شده،شماره n باشد و به همین ترتیب...\\
حال از انتها شروع می کنیم و حلقه اصلی فلوید وارشال را متناسب با اضافه شدن راس، ادامه می دهیم. یعنی اگر i راس اول تا کنون در گراف باشند و راس جدید i+1 اضافه شد، حلقه اصلی فلوید را به ازای 
\lr{k=i+1}
اجرا می کنیم. با این کار مقادیر 
\lr{$d_{u,v}$}
به ازای uوv کمتر از i+1 همواره مقدار صحیح خود را دارند زیرا همانند این است که الگوریتم فلوید را فقط برای i+1 راس اجرا می کردیم و اضافه عملیات های انجام شده تاثیری در این جواب ها ندارند.\\
حال به ازای هر درخواست کوتاه ترین مسیر، مقدار را در
\lr{O(1)}
از روی آرایه 
\lr{d[i][j]}
می خوانیم. کافی است در انتها به ترتیب عکس مقادیر خوانده شده را به عنوان خروجی بنویسیم.\\
اوردر الگوریتم برابر 
\lr{O($V^3 + Q$)}
است چون به طور سرشکن یکبار فلوید اجرا کرده ایم و دوبار هم لیست درخواست ها را خوانده ایم.\\

\pagebreak

\section*{پرسش ۲}
کافی است برای زوج راس
\lr{(u, v): $u!=v$}
یک راس در نظر بگیریم که نشان می دهد نفر اول و دوم هر کدام روی کدام راس هستند. حال هر راس گراف جدید
\lr{(u,v)}
 را به راس دیگر 
\lr{(u',v')} 
 وصل می کنیم اگر در گراف اصلی بین مولفه های اول این زوج یالی باشد و بین مولفه های دوم این دو زوج نیز یال باشد\\
 حال مسئله کمینه فاصله را از راس
 \lr{(s,e)}
 تا 
 \lr{(e,s)}
 پیدا می کنیم که با استفاده از هرم فیبوناچی و به طور سرشکن با روش دایکسترا
 در اوردر
 \lr{O(V lg(V) +E)}
 اجرا می شود که در اینجا ما 
 \lr{$V^2$}
 راس و 
 \lr{$E^2$}
 یال داریم\\
 تعداد یال های گراف ساخته شده برابر است با نصف مجموع درجات راس ها و در جه هر راس برابر حاصل ضرب درجه مولفه های آن است زیرا برای انتخاب یک راس که با این راس وصل باشد، می توان از مولفه اول به تعداد درجه آن راس جدید انتخاب کرد و برای مولفه دوم هم به همین صورت پس به تعداد ضربشان حالت جدید برای رفتن داریم.\\
 \lr{$E'=\Sigma{{d'}_i} = \Sigma{d_i\times d_j} \leq (\Sigma{d_i})^2 = E^2$}\\
 پس اوردر ما می شود:\\
 \lr{O($V^2 lg(V^2) + E^2$ = O($V^2 lg (V) + E^2$))}\\
 
\pagebreak
\section*{پرسش ۳}
گزاره را اثبات می کنیم که همواره حالت بدی وجود دارد که آن مقدار زمان لازم دارد.\\
n
راس می گیریم و به ترتیب به یکدیگر وصل می کنیم تا یک درخت یک بعدی بسازند.
و در هنگام ریلکس کردن از یال دورتر به راس مبدا( راسی که مقدار اولیه آن صفر است) شروع کنیم، در هر گام رئوس به فاصله i تا راس مبدا مقادیر نهایی خود را دارند و باقی رئوس مقدار بی نهایت دارند. پس باید n-1 گام این کار انجام شود پس به ازای n بزرگتر از 2:\\
\lr{$n>2: n-1 \geq \dfrac{n}{2} \Rightarrow t= \Omega(\dfrac{n}{2}) = \Omega(n) $}
\pagebreak

\section*{مراجع}




\end{document}
